% @TODO Change this
\chapter{Introduction}
\label{chapter1}

* ----- <roadsign> Peter Construction Co. </roadsign> *

% @TODO Do metheorologoy as an example

Computational Topology is an emerging field that leverages theory from topology to produce algorithm for solving various problems in structural biology, graphics, visualisation, medical imaging, X-ray crystallography and others. The mathematical field of topology is generaly interested in qualitative global properties of spaces. It is the natural field to study for example the "shape" of a surface or it's number of connected components. 

In this dissertation we will be most interested in utilizing computational topology in the field of scientific visualisation. There is an established tool in scientific visualisation called the contour tree. It is used to summarise the connectivity of planar cross sections of a surface. The power of the contour tree is that it can be used to identify the most topologically significant features in a data set with little to no human interaction. Such tools become invaluable when the amount of data that can be collected far exceeds the capability of humans to process manually. 

The central problem we will discuss in this dissertation is a theoretical computational efficiency limitation of the current state of the art algorithm for contour tree computation. We will demonstrate that problem we call a "W-Structure" is not only hindering algorithmic performance, but also shows up as a counter example to a connection that has been made between contour trees and a different field called Computational Homology called Computational Homology.

The material in this dissertation is spread throughout eight chapters. The second chapter provides the reader with the necessary mathematical background to tackle the rest of the chapters. Chapter three introduces contour tree and the newest parallel algorithm used to compute them. Chapter four explores the theoretical properties of the so-called W-Structures. In it we develop three algorithms for detecting and analysing them. Chapter five is devoted entirely to the subfield of Algebraic Topology called Homology. In chapter six we introduce a connection between contour trees and a tool from Homology called Persistent Homology. We use this connection to tackle a claim that was falsely made in a paper on the two. In the last chapter we present an empirical study on the w-structures by implementing and analysing all algorithms discussed in the dissertation. We use those algorithms to analyse real life data sets and demonstrate that these W-Structures do appear in practice and are a real problem.

%In this dissertation we pose the following questions



%This dissertation will focus on a theoretical problem in the field of computational topology.

%What is this dissertation about?
%What will you do in it exactly?
%How will you do it?
