% @TODO Change this
\chapter{Introduction}
\label{chapter1}

The mathematical field of topology studies the qualitative global properties of objects. It is the natural field to study for example the "shape" of a surface or the structure of its connected components. While topology has traditionally been only considered within pure mathematics new methods allow for topological properties of data to be computed in practice. These new methods are the study of Computational Topology \cite{topo-for-computing} and Topological Data Analysis \cite{topo-and-data}. These emerging fields on the edge of pure mathematics and computer science leverage theory from topology to produce algorithms for solving various problems in structural biology \cite{folded-molecules, hemo-globin},
visualisation \cite{topo-hierarchy, fiber-surfaces, flexible-isosurfaces}, medical imaging \cite{reeb-graph-brain} and
computer vision \cite{reeb-shape-analysis, structural-recognition}.

In this dissertation we will be most interested in utilizing computational topology in the context of scientific visualisation. We shall do so with the use of a tool that has been well established in recent years called the contour tree \cite{ct-big-paper}. The contour tree is a discrete graph data structure that is used to summarise the connectivity of planar cross sections of a surface. The utility of the contour tree is in that it can be used to identify and display the most topologically significant features in data with little to no human interaction. Such automated tools become invaluable when the amount of collected data far exceeds the capability of a human to process manually.

The central problem that we will discuss in this dissertation is a theoretical computational efficiency limitation of the current state of the art algorithm for data-parallel contour tree computation \cite{parallel-peak-pruning}. This issue is that certain substructures of the contour tree we call w-structures hinder parallel performance by serialising parts of the computation \cite{pathological-test-cases}. Our goal will be to understand how and why there w-strucures appear in contour trees of data. We will acomplish this by creating new algorithms that detect the existence of w-structures and extract them for further study.

The second problem we will address is that of contour tree simplification \cite{ct-branch-decomp}. This is the process of determining which parts of the contour tree correspond to the most topologically significant features of data. We will analyse the process of contour tree simplification through the lens of a different field of computational topology called persistent homology \cite{ph-a-survey}. We will pose and answear the question of whether the two approaches are equivalent. A counterexample based the w-structures we've introduced so far will make clear that fact that they are not.

The material in this dissertation is spread throughout eight chapters. The second chapter provides the reader with the necessary mathematical background to tackle the rest of the dissertation. Chapter three introduces the concept of contour trees and the state of the art algorithms found in the literature for computing them. Chapter four explores the theoretical properties the w-structures and develops three algorithms for detecting and analysing them. Chapter five is devoted entirely to introducing the subfield of Algebraic Topology called Homology. Chapter six will develop the tool for topological simplification called Persistent Homology and explore it's connection to contour tree simplification. In the last chapter we present an empirical study on the w-structures by implementing and analysing the algorithms we created in Chapter three of the dissertation. We will use those algorithms to analyse real life data sets and demonstrate that these w-structures do appear in them and that the algorithmic issues they cause translate to issues with practical computational performance.
