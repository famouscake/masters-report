\chapter{Chapter 1 Title}
\label{chapter1}

\section{Starting section}

\newtheorem{w-algorithm}{The W Detection Algorithm is Correct.}

\begin{proof}
After running the BFS twice we obtain two vertices $u$ and $v$ such that:
\begin{equation}
    \label{eq:su_all}
    w(s, u) \ge w(s, t), \forall t \in V(T)
\end{equation}

\begin{equation}
    \label{eq:uv_all}
    w(u, v) \ge w(u, t), \forall t \in V(T)
\end{equation}

Furthermore let $a$ and $b$ be two leaves that define a W diameter. Consequently $a$ and $b$ we know that:

\begin{equation}
    \label{eq:ab_all}
    w(a, b) \ge w(c, d), \forall c, d \in V(T)
\end{equation}

As $w(u, v) \le w(a, b)$, our end goal here is to bound $w(u, v)$ from bellow in terms of $w(a, b)$. Let also $t$ be the first vertex in the path between $a$ and $b$ that the first BFS starting at $s$ discovers. Clearly $t$ cannot be $a$ or $b$ unless $s$ is equal to $a$ or $b$.

This splits into several cases: \linebreak

{\em Case 1. When the path from $a$ to $b$ does not share any vertices with the path from $s$ to $u$.}

{\em Case 1.1. When the path from $u$ to $t$ goes through $s$.}

As $s \rightsquigarrow u$ is a subpath of $t \rightsquigarrow u$ then $w(t, u) \ge w(s, u)$. We also have that $w(s, u) \ge w(s, a)$ by equation \ref{eq:uv_all}. We can therefore conclude that $w(t, u) \ge w(s, a) \ge w(a, t)$ as $s \rightsquigarrow a$ is a subpath of $t \rightsquigarrow a$.

Now via path decomposition we have that:

$$ w(a, b) = w(b, t) + w(t, a) + x  $$
$$ w(u, b) = w(b, t) + w(t, u) + y .$$

Where $x, y \in \{0, 1\}$ depending on whether there is a kink at $t$ for the path from $a$ to $b$ and from $u$ to $b$ respectively. As $w(t, u) \ge w(a, t)$ we can show that:


$$ w(u, b) \ge w(b, t) + w(t, a) + y $$
$$ w(u, b) \ge w(b, t) + w(t, a) + x - x + y $$
$$ w(u, b) \ge w(a, b) - x + y $$
$$ w(u, b) \ge w(a, b) + (y - x) $$

But as $w(u, v) \ge w(u, b)$ (by equation \ref{eq:uv_all}) we obtain that:
$$ w(u, v) \ge w(a, b) + (y - x) $$

Considering all possible values that $x$ and $y$ can take, we can see that the minimum value for the right hand side of the equation is at $y = 0$ and $x = 1$. The final conclusion we may draw is that $w(u, v) \ge w(a, b) -1$.



%$a\rightsquigarrow b$

{\em Case 1.2. When the path from $u$ to $t$ does not go through $s$.}

If the path from $u$ to $t$ does not go through $s$ then the paths $s \rightsquigarrow t$ and $s \rightsquigarrow u$ have a common subpath. Let $s'$ be the last vertex in that subpath. We may wish to reduce this proof to the previous case, but it may be the case that the vertex $s'$ is a kink for one of the paths and not the other. We must account for that. We know that $w(t, u) \ge w(s', u)$ as a subpath. Through path decomposition we obtain that:

$$ w(s, a) = w(s, s') + w(s', a) + x $$
$$ w(s, u) = w(s, s') + w(s', u) + y $$

We know that $w(s, u) \ge w(s, a)$ and therefore:

$$ w(s, s') + w(s', u) + y \ge w(s, s') + w(s', a) + x  $$ 
$$ w(s', u) + y \ge w(s', a) + x $$ 
$$ w(s', u) \ge w(s', a) + (x - y)$$ 

Since $w(t, u) \ge w(s', u)$ we can further conclude that:

$$ w(t, u) \ge w(s', a) + (x - y)$$ 

From the fact that $t \rightsquigarrow a$ is a subpath of $s' \rightsquigarrow a$ it follows that $w(s', a) \ge w(t, a)$. This lets us obtain that:

$$ w(t, u) \ge w(t, a) + (x - y)$$ 

Now we are ready to proceed in a similar fashion as the previous case. We will decompose the paths from $b$ to $a$ and from $b$ to $u$ at the vertex $t$ as follows:

$$ w(b, a) = w(b, t) + w(t, a) + z  $$
$$ w(b, u) = w(b, t) + w(t, u) + w  $$
$$ w(b, u) \ge w(b, t) + w(t, a) + (x - y) + w $$
$$ w(b, u) \ge (w(b, t) + w(t, a) + z) - z + (x - y) + w $$
$$ w(b, u) \ge w(a, b) - z + (x - y) + w $$
$$ w(b, u) \ge w(a, b) + (x - y) + (w - z) $$

The minimum value for the right hand side of this equation is at $x, w = 0$ and $y, z = 1$. Now as $w(u, v) \ge w(u, b)$ we finally obtain that:

$$ w(u, v) \ge w(a, b) - 2 $$


{\em Case 2. When the path from $a$ to $b$ shares at least one vertex with the path from $s$ to $u$.}

% @TODO This is not complete!
%As we already know $w(s, u) \ge w(s, a)$. Furthermore $w(s, u) \ge w(t, u)$ and $w(s, a) \ge w(t, a)$ as they are subpaths of $s \rightsquigarrow u$ and $s \rightsquigarrow a$ respectively. Therefore $w(t, u) \ge w(t, a)$. We can now decompose the paths from $b$ to $a$ and from $b$ to $u$ at the vertex $t$ as follows:

We can do a path decomposition as follows:

$$ w(s, u) = w(s, t) + w(t, u) + x $$
$$ w(s, a) = w(s, t) + w(t, a) + y $$

As $w(s, u) \ge w(s, a)$ (by equation \ref{eq:uv_all})we obtain that:

$$ w(t, u) \ge w(t, a) + (y - x) $$

This again leads us to:

$$ w(b, a) = w(b, t) + w(t, a) + z  $$
$$ w(b, u) = w(b, t) + w(t, u) + w  $$
$$ w(b, u) \ge w(b, t) + w(t, a) + (x - y) + w $$
$$ w(b, u) \ge (w(b, t) + w(t, a) + z) - z + (x - y) + w $$
$$ w(b, u) \ge w(a, b) - z (x - y) + w $$
$$ w(b, u) \ge w(a, b) + (x - y) + (w - z) $$

Where similarly to the previous case we obtain that:

$$ w(u, v) \ge w(a, b) - 2 $$


Based on these cases we can conclude that for any input tree the algorithms will produce a path that is at most -2 away from the actual maximum path.

%$$ w(b, a) = w(b, t) + w(t, a) + x  $$
%$$ w(b, u) = w(b, t) + w(t, u) + y  $$



\end{proof}

\cite{parikh1980adaptive}
