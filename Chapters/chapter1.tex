% @TODO Change this
\chapter{Introduction}
\label{chapter2}

\section{Contour Trees}

\begin{lem} In a tree with no vertices of degree two at least half of the vertices are leaves. \end{lem}

\begin{proof}
    Let $T = (V, E)$ be a tree with no vertices of degree two and let $L \subseteq V$ be the set of all leaves. As all leaves have degree one we have that $L = \{u \in V: d(u) = 1\}$. Furthermore for any tree we know that $|E| = |V| - 1$. Let us now use the handshake lemma:

    $$ \sum_{u \in V}{d(u)} = 2|E| = 2(|V| - 1) = 2|V| - 2.$$

    We will not separe the sum on the leftmost hand side of the equation in two parts. One for the vertices vertices in $L$ and one for the vertices in $V\textbackslash L$.


    $$ \sum_{u \in L}{d(u)} + \sum_{u \in V\textbackslash L}{d(u)} = 2|V| - 2.$$

    All the vertices in $L$ are leaves. By definition the degree of a leaf is one. Therefore $\sum_{u \in L}{d(u)} = |L|$. This leads us to the following:

    $$  |L| + \sum_{u \in V\textbackslash L}{d(u)} = 2|V| - 2$$
    $$  |L|  = 2|V| - 2 - \sum_{u \in V\textbackslash L}{d(u)}.$$

    There are no vertices in $T$ of degree two and all vertices of degree one are in $L$. This means that all vertices in $V \textbackslash T$ have degree at least three. We can conclude that:
    $$\sum_{u \in V\textbackslash L}{d(u)} \ge \delta(T - L).|V\textbackslash L| = 3(|V| - |L|) $$

    Combining this with the previous equation we obtain that:

    $$  |L| \le 2|V| - 2 - 3(|V| - |L|)$$
    $$  |L| \le 2|V| - 2 - 3|V| + 3|L|$$
    $$  -2|L| \le -|V| - 2$$
    $$  |L| \ge \frac{|V|}{2} + 1$$

    Which is exactly what we set out to proove.


\end{proof}

For future reference we would also like to present a claim that is more general than this. Notice that we could have required that $T$ has no vertices of degree less than $n \in \{3, 4, 5, ...\}$. If we make the substitution accordingly we obain that:

    $$\sum_{u \in V\textbackslash L}{d(u)} \ge n(|V| - |L|) $$

    $$  |L| \ge \frac{n - 2}{n - 1}|V| + \frac{2}{n - 1}$$

    As $n$ gets larger we have that $ \lim_{n \to \infty}\frac{2}{n - 1} = 0$ and by L'Hopital's rule:
    
    $$\lim_{n \to \infty} \frac{n - 2}{n - 1} = \lim_{x \to \infty} \frac{(n - 2)'}{(n - 1)'} = 1$$

    This means that for sufficiently large $n$ almost all of the vertices in a tree are leaves. Even for $n = 11$ we already have that at least $8/10$ of the vertices are leaves. This result will come in handy in one of the proofs in the second chapter.
 
