\chapter{Introduction}
\label{chapter1}

The mathematical field of topology studies the qualitative properties of geometric objects. It is the natural field to study for example the shape of a surface by decomposing it into path-connected components. While topology has traditionally been only considered within pure mathematics new methods allow for topological properties of data to be computed in practice. These new methods are the study of Computational Topology \cite{topo-for-computing} and Topological Data Analysis \cite{topo-and-data}. These emerging fields on the edge of pure mathematics and computer science leverage theory from topology to produce algorithms for solving various problems in structural biology \cite{folded-molecules, hemo-globin}, visualisation \cite{topo-hierarchy, fiber-surfaces, flexible-isosurfaces}, medical imaging \cite{reeb-graph-brain} and
computer vision \cite{reeb-shape-analysis, structural-recognition}.

In this dissertation we will be most interested in utilizing computational topology in the context of scientific visualisation. We shall do so with the use of a tool that has been well established in recent years called the contour tree \cite{ct-big-paper}. The contour tree is a discrete graph data structure that is used to summarise the connectivity of planar cross sections of a scalar field. The utility of the contour tree is in that it can be used to identify and display the most topologically significant features in data with little to no human interaction. Such automated tools become invaluable when the amount of collected data far exceeds the capability of a human to process manually.

The central problem that we will discuss in this dissertation is a theoretical computational efficiency limitation of the current state of the art algorithm for data-parallel contour tree computation \cite{parallel-peak-pruning}. The cause of this issue are certain substructures of contour trees we call w-structures. They hinder parallel performance by serialising parts of the computation \cite{pathological-test-cases}. Our goal is it to understand how and why these w-structures appear in contour trees of data. We will accomplish this by developing new algorithms that detect the existence of w-structures and extract them for further study.

The second problem we will address is that of contour tree simplification \cite{ct-branch-decomp}. Contour tree simplification is the process of reducing the size of a contour tree by removing parts of it that correspond to topologically insignificant  features of data. We will analyse the process of contour tree simplification using a more general tool from topological data analysis called Persistent Homology \cite{ph-a-survey}. We will pose and answer the question of whether the two approaches produce equivalent results. A counterexample based the w-structures we have introduced will make clear that fact that they do not.

The material in this dissertation is spread throughout eight chapters. The second chapter provides the reader with the necessary mathematical background to tackle most of the rest of the dissertation. Chapter three introduces the concept of contour trees and the state of the art algorithms found in the literature for computing them. Chapter four constitutes the first part of our original research. We explore the theoretical properties of w-structures and develop three algorithms for detecting and extracting them from contour trees. In chapter five we take a step back to introduce more mathematical background that would enable us to address our second goal of comparing contour tree simplification and persistent homology. In this chapter we will cover the basics of a subfield of Algebraic Topology called Homology and introduce the theory behind Persistent Homology. Chapter six is the second part of our original research. We explore the connection between contour tree simplification and persistent homology by computing and comparing the output of both on a specifically chosen data set. In chapter seven we present an empirical study on the w-structures by implementing and analysing the algorithms we created in chapter four and running them on both artificially generated and on real life data sets. We will use those algorithms to demonstrate that the w-structures do appear in real life data and that the theoretical algorithmic issues they cause translate to issues with practical computational performance. Chapter eight is devoted to the conclusion, personal reflection and our thoughts on future directions.
