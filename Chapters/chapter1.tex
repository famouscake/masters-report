% @TODO Change this
\chapter{Something "W" This Way Comes!}
\label{chapter1}

\section{What is a W in a height graph?}

A height graph is a graph $G = (V, E)$ together with a real valued function $h$ defined on the vertices of $G$. In the case of heights graphs that are used in conjunction with Morse functions we will also require that all vertices have unique heights. In other words $h(u) \ne h(v)$ for all $u ,v \in V(G)$ where $u \ne v$. The function $h$ naturally introduces an ordering of the vertices. From now on for we will consider the vertices of $G$ in ascending order. That is to say $V(G) = \{v_1, v_2, ... , v_n\}$ where $h(v_1) < h(v_2) < ... < h(v_n)$. This lets us work with the indices of the vertices without having to constantly compare their heights. In this notation $h(v_i) < h(v_j)$ when $i < j$.
%@TODO Remove this if you will not use it.

Introducing the height function allows us to talk about ascending and descending paths. A path in the graph $(u_1, u_2, ... , u_k)$ where $u_i \in V(G)$ for $i \in \{1, 2, ..., k\}$ and $u_iu_{i+1} \in E(G)$ for $i \in \{1, 2, ..., k-1\}$ is ascending whenever $h(u_1) < h(u_2) < ... < h(u_k)$. Conversely if we traverse the path in the opposite direction it would be descending. Thus we will call these path monotone to avoid having to commit to a specific direction of travel.

A kink in a path is a vertex whose two neighbours are either both higher or both lower. More formally given the path $(u_1, u_2, ... , u_k)$ a vertex $u_i \ne u_1, u_k$ is a kink if $h(u_i) \ne \big( min(h(u_{i-1}),~h(u_{i+1}),~max(h(u_{i-1}),~h(u_{i+1}) \big)$. To avoid cumbersome notation in this context we shall adopt a slight abuse of notation and in the future write the previous statement as $h(u_i) \ne $ or $ \in $ the interval $\big(h(u_{i-1}),~h(u_{i+1}) \big)$ ever if $h(u_{i-1}) > h(u_{i+1})$.

Intuitively we can think of the number of kinks in a path as a metric that is defined on the path. Analogous to how the length of a path is the number of edges in that path we shall define the w-length of a path as the number of kinks in that path. Indeed let us denote any path from $u$ to $v$ as $u \rightsquigarrow v$ and by $d(u \rightsquigarrow v)$ measure the length of the longest path between $u$ and $v$ and by $w(u \rightsquigarrow v)$ the path with the largest number of kinks. One immediate obversation we can make is that $d(u \rightsquigarrow v) > w(u \rightsquigarrow v)$. This follows from the fact that a path with $k$ vertices has length $k-1$ and at most $k-2$ internal vertices which may be kinks.


\section{W Diameter Detector}


The algorithm for computing the "near" to maximum W in a given height tree is based on an algorithm for finding the longest path in tree - it's diameter. It consist of running a standard Breadth First Search (BFS) from any vertex in the tree and recording the leaf that is furthest from that vertex. It can be show [ref] that the furthest leaf is a starting vertex of a diameter of the tree. To obtain the actual diameter then one can run another BFS rooted at that leaf.

Similarly this new algorithm runs a BFS from any vertex in the graph and records the leaf that is farthest in terms of w-length (or number of kinks on the path). This furthest leaf if guaranteed to be either the endpoint of W diameter of the tree or some path that has w-length at least $-2$ of the actual W diameter of the tree. 

The algorithms works as follows:


\begin{algorithm}
\caption{Computing the W Diameter of a Height Tree.}

\begin{algorithmic}[1]

\Function{W\_BFS}{T, root}
    \State root.d = 0
    \State root.$\pi$ = root
    \State furthest = root

    \State Q = $\emptyset$
    \State Enqueue(Q, root)

    \While {Q $\ne \emptyset$}
        \State u = Dequeue(Q)

        \If {u.d $\ge$ furthest.d}
            \State furthest = u
        \EndIf

        \ForAll {v $\in$ T.$Adj$[u]} 
            \If {v.$\pi$ == $\emptyset$}
                \State v.$\pi$ = u
                \If {u $\notin$ \big(v,~u.$\pi$\big)}
                    \State v.d = u.d + 1
                \Else
                    \State v.d = u.d
                \EndIf

                \State Enqueue(Q, v)

            \EndIf
        \EndFor
    \EndWhile
    \State Return furthest
\EndFunction

\Function{Calculate\_W\_Diameter}{T}
    \State s = <any vertex>
    \State u = W\_BFS(T, s)
    \State v = W\_BFS(T, u)
    \State return b.d
\EndFunction

\end{algorithmic}
\end{algorithm}

%\newtheorem{w-algorithm}{The W Detection Algorithm is Correct.}
Before proving the correctness of the algorithm we must first establish two useful properties that related the w length of to it's subpaths.
Let $a \rightsquigarrow b$ be a path from $a$ to $b$ in the tree. The we will denote as $w(a, b)$ the number of kinks in that path (which is unique in the tree).

\begin{defn} Subpath Property  \end{defn}

aLet $a \rightsquigarrow b$ be a subpath of $c \rightsquigarrow d$. Then $w(a, b) \le w(c, d)$. 

This property follows from the fact that all kinks of the path from $a$ to $b$ are also kinks of the path from $c$ to $d$.

\begin{defn} Path Decomposition Property Property  \end{defn}

Let $a \rightsquigarrow b$ be a path and $t$ be a vertex on that path. Then $w(a, b) = w(a, t) + w(t, b) + x$, where $x \in \{0 ,1\}$ depending on whether $t$ is a kink in the path from $a$ to $b$. 

Indeed $t$ can be a kink in the path from $a$ to $b$, but it cannot be a kink in the paths from $a$ to $t$ and from $t$ to $b$ because it is an endpoint of both. All other kinks are preserved in either $w(a, t)$ or $w(t, b)$. To account for this we must consider additional cases for both possible values of $x$ when using this property.



\begin{lem} The Algorithm produces the endpoints of a path who is at most 2 kinks shy of being the kinkiest path in the tree. \end{lem}


\begin{proof}
After running the BFS twice we obtain two vertices $u$ and $v$ such that:

\begin{equation}
    \label{eq:su_all}
    w(s, u) \ge w(s, t), \forall t \in V(T)
\end{equation}

\begin{equation}
    \label{eq:uv_all}
    w(u, v) \ge w(u, t), \forall t \in V(T)
\end{equation}

Furthermore let $a$ and $b$ be two leaves that are the endpoints of a path that is a W diameter. For any such pair we know that:

\begin{equation}
    \label{eq:ab_all}
    w(a, b) \ge w(c, d), \forall c, d \in V(T)
\end{equation}

By this equation we have that $w(a, b) \ge w(u, v)$. Our goal in this proof will be to give a formal lower bound on $w(u, v)$ terms of $w(a, b)$. To this end let $t$ be the first vertex in the path between $a$ and $b$ that the first BFS starting at $s$ discovers. From this description it is clear that $t$ cannot be $a$ or $b$ unless $s$ is equal to $a$ or $b$.

The proof can then be split into several cases. \linebreak

{\em Case 1. When the path from $a$ to $b$ does not share any vertices with the path from $s$ to $u$.}

{\em Case 1.1. When the path from $u$ to $t$ goes through $s$.}

In this case $s \rightsquigarrow u$ is a subpath of $t \rightsquigarrow u$, which in turn means that $w(t, u) \ge w(s, u)$. By equation \ref{eq:uv_all} we also have that $w(s, u) \ge w(s, a)$. We can therefore conclude that $w(t, u) \ge w(a, t)$ as $s \rightsquigarrow a$ is a subpath of $t \rightsquigarrow a$.

Now via path decomposition of $a \rightsquigarrow b$ and $u \rightsquigarrow b$ at $t$ have that:

$$ w(a, b) = w(b, t) + w(t, a) + x  $$
$$ w(u, b) = w(b, t) + w(t, u) + y .$$

Where $x, y \in \{0, 1\}$ depending on whether there is a kink at $t$ for the path from $a$ to $b$ and from $u$ to $b$ respectively. As $w(t, u) \ge w(a, t)$ we can show that:


$$ w(u, b) \ge w(b, t) + w(t, a) + y $$
$$ w(u, b) \ge w(b, t) + w(t, a) + x - x + y $$
$$ w(u, b) \ge w(a, b) - x + y $$
$$ w(u, b) \ge w(a, b) + (y - x) $$

But as $w(u, v) \ge w(u, b)$ (by equation \ref{eq:uv_all}) we obtain that:
$$ w(u, v) \ge w(a, b) + (y - x) $$

Considering all possible values that $x$ and $y$ can take, we can see that the minimum value for the right hand side of the equation is at $y = 0$ and $x = 1$. The final conclusion we may draw is that $w(u, v) \ge w(a, b) -1$.



%$a\rightsquigarrow b$

{\em Case 1.2. When the path from $u$ to $t$ does not go through $s$.}

If the path from $u$ to $t$ does not go through $s$ then the paths $s \rightsquigarrow t$ and $s \rightsquigarrow u$ have a common subpath. Let $s'$ be the last vertex in that subpath. We will be able to reduce this case to the previous one by using $s'$ in the place of $s$. We must only account for a situation where $s'$ is a kink for one of the paths and not the other. We know that $w(t, u) \ge w(s', u)$ (as a subpath) and through path decomposition we obtain that:

$$ w(s, a) = w(s, s') + w(s', a) + x $$
$$ w(s, u) = w(s, s') + w(s', u) + y $$

We know that $w(s, u) \ge w(s, a)$ and therefore:

$$ w(s, s') + w(s', u) + y \ge w(s, s') + w(s', a) + x  $$ 
$$ w(s', u) + y \ge w(s', a) + x $$ 
$$ w(s', u) \ge w(s', a) + (x - y)$$ 

Since $w(t, u) \ge w(s', u)$ we can further conclude that:

$$ w(t, u) \ge w(s', a) + (x - y)$$ 

From the fact that $t \rightsquigarrow a$ is a subpath of $s' \rightsquigarrow a$ it follows that $w(s', a) \ge w(t, a)$. This lets us obtain that:

$$ w(t, u) \ge w(t, a) + (x - y)$$ 

Now we are ready to proceed in a similar fashion as the previous case. We will decompose the paths from $b$ to $a$ and from $b$ to $u$ at the vertex $t$ as follows:

$$ w(b, a) = w(b, t) + w(t, a) + z  $$
$$ w(b, u) = w(b, t) + w(t, u) + w  $$
$$ w(b, u) \ge w(b, t) + w(t, a) + (x - y) + w $$
$$ w(b, u) \ge w(b, t) + w(t, a) + z - z + (x - y) + w $$
$$ w(b, u) \ge w(a, b) - z + (x - y) + w $$
$$ w(b, u) \ge w(a, b) + (x - y) + (w - z) $$

The minimum value for the right hand side of this equation is at $x, w = 0$ and $y, z = 1$. Now as $w(u, v) \ge w(u, b)$ we finally obtain that:

$$ w(u, v) \ge w(a, b) - 2 $$


{\em Case 2. When the path from $a$ to $b$ shares at least one vertex with the path from $s$ to $u$.}

% @TODO This is not complete!
%As we already know $w(s, u) \ge w(s, a)$. Furthermore $w(s, u) \ge w(t, u)$ and $w(s, a) \ge w(t, a)$ as they are subpaths of $s \rightsquigarrow u$ and $s \rightsquigarrow a$ respectively. Therefore $w(t, u) \ge w(t, a)$. We can now decompose the paths from $b$ to $a$ and from $b$ to $u$ at the vertex $t$ as follows:

We can do a path decomposition as follows:

$$ w(s, u) = w(s, t) + w(t, u) + x $$
$$ w(s, a) = w(s, t) + w(t, a) + y $$

As $w(s, u) \ge w(s, a)$ (by equation \ref{eq:uv_all})we obtain that:

$$ w(t, u) \ge w(t, a) + (y - x) $$

This again leads us to:

$$ w(b, a) = w(b, t) + w(t, a) + z  $$
$$ w(b, u) = w(b, t) + w(t, u) + w  $$
$$ w(b, u) \ge w(b, t) + w(t, a) + (x - y) + w $$
$$ w(b, u) \ge (w(b, t) + w(t, a) + z) - z + (x - y) + w $$
$$ w(b, u) \ge w(a, b) - z (x - y) + w $$
$$ w(b, u) \ge w(a, b) + (x - y) + (w - z) $$

Where similarly to the previous case we obtain that:

$$ w(u, v) \ge w(a, b) - 2 $$


Based on these cases we can conclude that for any input tree the algorithm will produce a path that is at most -2 away from the actual maximum path.

%$$ w(b, a) = w(b, t) + w(t, a) + x  $$
%$$ w(b, u) = w(b, t) + w(t, u) + y  $$



\end{proof}

\cite{parikh1980adaptive}
