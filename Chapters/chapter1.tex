% @TODO Change this
\chapter{Introduction}
\label{chapter1}

* ----- <roadsign> Peter Construction Co. </roadsign> *

% @TODO Do metheorologoy as an example

Computational Topology is an emerging field that leverages theory from topology to produce algorithms for solving various problems in structural biology, graphics, visualisation, medical imaging, X-ray crystallography and others. The mathematical field of topology is largely interested in qualitative global properties of objects. It is the natural field to study for example the "shape" of a surface or it's number of connected components. 

In this dissertation we will be most interested in utilizing computational topology in the field of scientific visualisation. We shall do so through a tool that has been well established in recent years in scientific visualisation called the contour tree. The contour tree is a discrete graph data structure that is used to summarise the connectivity of * planar cross sections of a surface *. The usefulness of the contour tree is in that it can be used to identify the most topologically significant features that arise in data with little to no human interaction. Such tools become invaluable when the amount of data that can be collected far exceeds the capability of humans to process manually. 

The central problem that we will discuss in this dissertation is a theoretical computational efficiency limitation of the current state of the art algorithm for contour tree computation. The issue current parallel contour tree algorithms have is that certain substructures we call W-Structures of the contour tree slow down computation by serialising it along these W-Structures. They are in a sense the critical path of the parallel computation and we would like to analyse them theoretically to find out why that is. We will also use the W-Structures in another theoretical setting of Persistent Homology as a counter example to a claim that has been made on the connection of contour tree computation and persistent homology computation.

The material in this dissertation is spread throughout eight chapters. The second chapter provides the reader with the necessary mathematical background to tackle the rest of the dissertation. Chapter three introduces the concept of contour trees and the state of the art parallel algorithm we have for computing them. Chapter four explores the theoretical properties of the critical substructures of contour trees we defined as W-Structures. We will develop three algorithms for detecting and analysing them. Chapter five is devoted entirely to the subfield of Algebraic Topology called Homology. In chapter six we introduce a connection between contour trees and a tool from Computational Homology called Persistent Homology. We use this connection to tackle a claim that was falsely made in a paper that states there is a connection between the two. In the last chapter we present an empirical study on the w-structures by implementing and analysing all algorithms discussed in the dissertation. We use those algorithms to analyse real life data sets and demonstrate that these W-Structures do appear in practice and are a real problem.

%In this dissertation we pose the following questions



%This dissertation will focus on a theoretical problem in the field of computational topology.

%What is this dissertation about?
%What will you do in it exactly?
%How will you do it?
