% I have come here to chew bubble gum and kick ass. And I'm all out of bubble gum.

As the sheer amount of data that is collected in business and scientific applications reaches exascale it is becoming an increasingly challenging task to process it and extract useful information. To do so we not only need efficient algorithms that work over distributed and multicore hardware but also ways of reducing its volume.

Topological Data Analysis is a general framework for analysing and comparing the significance of subsets of data. This allows us to process only the most significant ones and greatly reduce computational overhead. Developing algorithms in this field is a challenging task because of the ballancing act one must do between continous mathematical models and discrete computational models.

A principal tool in topological data analysis that is used extensively in scientific visualisation is the contour tree. It is a discrete data structure that represents the topological structure of a scalar field. The aim of this dissertation is to lay the foundations of understanding a particular pathological edge case that emerges in the state of the art parallel algorithm for contour tree computation and to explore its connection to other fields of topological data analysis.
